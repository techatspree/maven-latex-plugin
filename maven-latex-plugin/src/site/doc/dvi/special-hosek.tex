%%%BC  Don, I have suggested some editing to your fine epistle.
%%%BC  They are all preceeded by those five letter characters, but not
%%%BC  not always starting in column 1.  I would like to distribute
%%%BC  the edited form to our printer standards committee along with
%%%BC  Tom Reid and Glen V.'s work. Do I have your permission to
%%%BC  make those corrections?  The opinions are really notes to me.
%%%BC  These notes show a lot of thought.
%%%BC  Thanks, Bart Childs.
%%%BC
%%% Format=LaTeX
%%%
\documentstyle[twoside]{article}
                             % I'm not wedded to this.
                             % LaTeX was chosen so that the recipient
                             % could format this to match whatever style
                             % they like.

\font\logo=logo10    % Ideally this would be logo10
\def\AmSTeX{$\cal A\kern-.1667em\lower.5ex\hbox{$\cal M$}\kern-.075em
  S$-\TeX}
\def\BibTeX{{\rm B\kern-.05em{\sc i\kern-.025em b}\kern-.08em
    T\kern-.1667em\lower.7ex\hbox{E}\kern-.125emX}}
\def\SliTeX{{\rm S\kern-.06em{\sc l\kern-.035emi}\kern-.06em T\kern
   -.1667em\lower.7ex\hbox{E}\kern-.125emX}}
\def\TeXMaG{T\kern-.1667em\lower.5ex\hbox{E}\kern-.125emX\-% TeX part
   M\kern-.1667em\lower.5ex\hbox{A}\kern-.2267emG}
\def\TeXhax{T\kern-.1667em\lower.5ex\hbox{E}\kern-.125emX\-{\sc hax}}
\def\MF{\hbox{\logo META}\-\hbox{\logo FONT}}
\def\PS{{\sc PostScript}}
\def\Unix{{\sc Unix}}

\chardef\BS=`\\    % This is a better way of doing things than Lamport's
                   % \newcommand

\newcommand{\Special}{{\tt\BS special}}
\newcommand{\DVI}{{\tt DVI}}

\def\.#1{\ifvmode\leavevmode\fi\hbox{\tt\let\\=\BS
         \chardef\{=`\{\chardef\}=`\}\chardef\ ='40
         \chardef\&=`\&#1}}

\begingroup
\catcode`\_=\active\catcode`\-=\active\catcode`\|=\active
\gdef\BNFdefs{\def\&##1{\hbox{\bf##1}}
   \def\dash{\char'174\space} % '174 is an em-dash.
   \def\e{\hbox to 1em{\hss$\epsilon$\hss}}
   \catcode`\_=\active\let_=\_
   \catcode`\-=\active\def->{&\hbox to 1em{\hss$\rightarrow$\hss}}
      \chardef\-=`\-
   \catcode`\|=\active\def|{&\hbox to 1em{\hss$\mid$\hss}}
      \def\|{\hbox to 1em{\hss$\mid$\hss}}
   \let\\=\cr}
\endgroup

\def\BNF{\bgroup\BNFdefs\halign\bgroup\hss\it\boldmath##
         &\it\boldmath##\hfil\cr}
\def\endBNF{\crcr\egroup\egroup\noindent}

%%%
%%% Other definitions should be placed here:


%%%
%%% Here are some LaTeX commands for the title
\title{Proposed \DVI\ \Special\ command standard}
\author{Don Hosek\\
         Harvey Mudd College\thanks{BITNET: {\tt DHOSEK@HMCVAX}.
         Phone: (714)~621-8000.
         Postal address: Platt Campus Center, Harvey Mudd College,
         Claremont, CA 91711}\\
         Claremont, CA}
%%%BC I like the telephone better. Mine is 409-845-5470
%%%DH There's little hope of catching me on the phone, but I'll include it
%%%DH any way.
\date{August 6, 1987}

\begin{document}
\maketitle

The information in this report includes some ideas based on responses to
an article sent to the \TeXhax\ mailing list and printed in the \TeXMaG\
electronic ``magazine''. Among those people who sent me suggestions were
Neta Amit, Stephan Bechtolsheim, Peter Breitenlohner, `dorab', John
Gourlay, Jean-Francois Lamy, Kevin LaRue, William LeFebvre, and Thomas
J.~Reid. Also, Dean Guenther invited me to send my suggestions for the
TUG meeting. I apologize if I left anyone out of this list.

\section{Introduction}
It is becoming clear that some sort of standard on the syntax and
effect of \Special\ commands is necessary. In my original report to
\TeXhax\ and \TeXMaG\ \cite{hosek:specials}, I outlined some of the
issues to be confronted in the development of a \Special\ standard.
These included device dependency, naming conventions and syntax,
graphics, and scope (see also \cite{fox:changebars}).

Dean Guenther informs me that he hopes to see some definite standards
for \Special\ commands set after this year's TUG meeting, and
Stephan Bechtolsheim (\.{dvi2ps}) is holding off on writing the
\Special\ handling code for his latest version of the driver until the
\Special\ command issue is resolved. The suggestions in this paper are
just that, suggestions; the specifications of this proposed standard are
open to change. The final standard that comes out of this meeting should
be specific enough that all implementations of it can generate identical
results, but also open to extension to reflect changes and innovations
in printing technology.

\section{Syntax}
The general syntax for a \Special\ command is: optional environment
setting commands followed by the special command with multiple special
commands separated by semicolons. A Backus-Naur Form\footnote{See
\cite{aho:compilers} for additional information on BNF} syntax for
\Special\ commands follows:
\begin{BNF}
special_command-> spec_cmd \.; special_command \| spec_cmd\\
spec_cmd-> environment_setting_cmds cmd\\
         | \e\\
environment_setting_cmds-> env_cmd environment_setting_cmds\\
         | \e\\
\end{BNF}
Most possible {\it cmd\/} or {\it env\_cmd} values accept options
that have the syntax described below:
\begin{BNF}
options-> required_space opt options \| \e\\
required_space-> \.\ \ optional_space\\
optional_space-> required_space \| \e\\
opt-> angle \| position \| scaling \| tag \| length \| color\\
         | cap_shape \| file_name\\
angle-> \." real \| \./ real \., real\\
position-> \.( dimen \., dimen \.) \| \.+ \.( dimen \., dimen \.)\\
scaling-> \.< opt_real \., opt_real \.> \| \.[ dimen \., dimen \.]\\
tag-> chars \.:\\
length-> req_dimen\\
color-> \.{black} \| \.{white}\\
         | \.{red} \| \.{blue} \| \.{yellow}\\
         | \.{violet} \| \.{green} \| \.{orange}\\
         | \.{brown}\\
         | color \.{\&} color\\
cap_shape-> \.{butt} \| \.{round} \| \.{square} \| \.{bar}\\
file_name-> opt_area name opt_ext\\
real-> digit digit_string\\
         | digit digit_string \.. digit_string\\
opt_real-> real \| \e\\
digit-> \.0 \| \.1 \| \.2 \| \.3 \| \.4 \| \.5 \| \.6 \| \.7 \| \.8 \| \.9\\
digit_string-> digit digit_string \| \e\\
dimen-> opt_real unit\\
req_dimen-> real unit\\
unit-> scaled_unit \| \.{true} scaled_unit \| \e\\
scaled_unit-> \.{pt} \| \.{pc} \| \.{in} \| \.{bp} \| \.{cm} \| \.{mm}
         \| \.{dd} \| \.{cc} \| \.{sp}\\
chars-> chr chars \| \e\\
chr-> letter \| digit \\
letter-> \.A \dash \.Z \| \.a \dash \.z\\
opt_area-> \.< name \.> \| name \.: \| \e\\
opt_ext-> \.. name \| \e\\
\end{BNF}

Details of the semantics of the various possiblities for {\it env\_cmd\/}
or {\it cmd\/} are given in sections~\ref{sec:env} and~\ref{sec:cmds}.
However, a brief overview of the various {\it env\_cmd\/} and {\it cmd\/}
values will be given, followed by  an explanation of the various
{\it option\/} values.

\subsection{Overview of {\it env\_cmd}}
An environment command or commands affect {\it only\/} the first
{\it cmd\/} following it. Since {\it cmd\/} is never empty, this means
that an environment setting command can not affect a {\it cmd\/} that is
in a different \Special\ than the one in which the {\it env\_cmd\/}
appeared.

The following is a list of possible {\it env\_cmd\/} values with
option list and a brief description. A detailed explanation will be
deferred until Section~\ref{sec:env}.
\begin{description}
\item[\.{black} {\it color\/}] Set the color to be used as black.
\item[\.{linecap} {\it cap\_shape\/}] Set the shape of line ends.
\item[\.{linewidth} {\it length\/}] Set the width of lines.
\item[\.{offset} {\it position\/}] Change the location of the output
         from {\it cmd.\/}
\item[\.{turn} {\it angle\/}] Rotate the command.
\item[\.{scale} {\it scaling\/}] Scale the command.
\item[\.{white} {\it color\/}] Set the color to be used as white.
\end{description}

\subsection{Overview of {\it cmd}}
\begin{description}
\item[\.{arc} {\it length\/} {\it angle\/} {\it angle\/}] Draw an arc.
\item[\.{beginbar} {\it tag\/}] Begin a vertical bar at the current \DVI\
         $h,v$ coordinates.
\item[\.{circle} {\it length}] Draw a circle.
\item[\.{disk} {\it length\/}] Draw a filled-in circle.
\item[\.{endbar} {\it tag\/}] End a vertical bar at the current \DVI\ $v$
         coordinate.
\item[\.{include} {\it tag\/} {\it file\_name\/}] Include the graphics
         image of the type {\it tag\/} contained in {\it file\_name.}
\item[\.{landscape}] Orient the page in the landscape position.
\item[\.{line} {\it length\/} {\it angle\/}] Draw a line.
\item[\.{message}] Send {\it option\/} to the terminal.
\item[\.{newsheet}] Force a new sheet.
\item[\.{pie} {\it length\/} {\it angle\/} {\it angle\/}] Draw a
         filled-in arc.
\item[\.{portrait}] Orient the page in the portrait position.
\item[\.{rotate} {\it angle\/}] Rotate the page about the center of the
         {\sl real\/} page.
\item[\.{spline} {\it position\/} {\it position\/} {\it position\/}] Draw
         a B\'ezier cubic.
\item[\.{textcolor} {\it color\/}] Set the current text color.
\end{description}

\subsection{Overview of {\it opt}}
The order in which the {\it opt\/} values appear in {\it options\/}
is not important unless more than one {\it opt\/} of a given type
appears in the paramter list.

\subsubsection{Angles}\label{sec:angle}
An angle may be specified in two manners. It may be given as~\." followed
by a real number (let us call it~$n$). The angle~$\theta$ represented,
is then considered to be (in degrees) $n \bmod 360$ if
$n \bmod 360\le180$, and $360-(n \bmod 360)$ otherwise. It may also
be given as \./ followed by two reals separated with a comma (let us
call them $m$~and~$n$). If $n>0$ and $m\ge0$ then
$\theta=\arctan\left|m/n\right|$; if $n>0$ and $m<0$ then
$\theta=180-\arctan\left|m/n\right|$; if $n<0$ and $m\ge0$ then
$\theta=-180+\arctan\left|m/n\right|$; if $n<0$ and $m<0$ then
$\theta=-\arctan\left|m/n\right|$; if $n=0$ and $m<0$ then $\theta=180$;
and if $n=0$ and $m>0$ then $\theta=0$. If $m=0$ and $n=0$ then the
command should be flushed.
All angles are considered to be counterclockwise measured from ``up''
on the current page orientation, unless specified otherwise.

If an angle must be rounded,\footnote{This generally only applies to
rotation of figures or pages where the array of available rotations
would tend to be relatively small, such as [-180, -90, 0, 90].} it should
be rounded as follows:
\begin{enumerate}
\item Assuming that the available angles of rotation are in the
         array~$p$, scan through~$p$ from $1$~to $\it last$ in increasing
         order. Continue to scan until $p_i\ge\theta$.
%%%BC I think that $p_i\ge\theta$. is more direct
%%%DH Good point.
\item If $p_i-\theta\le\theta-p_{i-1}$ then round $\theta$ to $p_i$,
         otherwise, round $\theta$ to $p_{i-1}$ ($p_0$ should be set
         ridiculously low, to make sure that $\theta$ is rounded to $p_1$
         if $\theta<p_1$).
\end{enumerate}

\subsubsection{Dimensions}
Dimensions are specified in essentially the same format as \TeX\ uses
(see \cite[Chapter~10]{knuth:texbook} for details on the meanings of
the units used). One extension to this format is that integer values
shall be assumed to be in terms of \DVI~units (as defined in
\cite[\S13--20]{ware:dvitype}). In the case of \TeX82, \DVI~units are
always scaled points ($2^{-16}$~points) so a construction such as
\verb|\number\mydimen| will yield a valid dimension for a \Special\
command. This could be used by macros that generate \Special\ commands
rather than \verb|\the\mydimen| to keep computations by the postprocesor
down.

Dimensions should be rounded to pixel dimensions using the algorithim
of {\it rule\_pixels} \cite[\S76]{ware:dvitype}.

\subsubsection{Positions}
An offset may be described in two manners. An absolute offset is given
as $\.(x,y\.)$ where $x$~and~$y$ are optional dimensions. If the
dimension is omitted, the current $h$~or %%%BC inserted a blank here
    $v$ is substituted for it. The
standard \DVI\ coordinate system is used (see \cite[\S14]{ware:dvitype}).

A relative offset is specified as $\.+\.(x,y\.)$ where $x$~and %%%BC and here
    $y$ are
optional dimensions. If a dimension is omitted, it is treated as zero.
$x$~is added to the current~$h$, and $y$~is added to the current~$v$
to get the position for the output of the \Special\ command.

\subsubsection{Scaling}
Scaling may be described in two manners. Absolute scaling is given as
$\.[x,y\.]$ where $x$ and $y$ are optional dimensions. If a dimension
is omitted, the natural width of the \Special's output is substituted
for it. When given, $x$~or~$y$ cause the width or height of the
\Special's output to be scaled to match the given dimension.

Relative scaling is specified as $\.< x, y \.>$ where $x$~and~$y$ are
optional reals. If $x$~or~$y$ are omitted, then they are treated as~1.
Scaling is performed by multiplying the natural width of the special
by~$x$ and the natural height by~$y$.

\subsubsection{Tags}
A tag is used to indicate the nature of external information to be
included. Their interpretation is given under the description of
each command that uses them in section~\ref{sec:cmds}. Tags may be of
any length.
%%%BC When I first read this I could not figure out what it was.
%%%BC Then when I did, I became absolutely opposed. Having a TAG
%%%BC   means that it cannot be device independent. This should be
%%%BC   left as it is for input because the thoughts must be considered.
%%%DH Not necessarily device dependant; the tag only indicates what the
%%%DH   information comes from (they are only used by the graphics inclusion
%%%DH   facility at present) Just because I'm driving a PS printer doesn't
%%%DH   mean that I can't or shouldn't attempt to deal with a Tektronix plot
%%%DH   file or a series of QUIC commands or whatever.

\subsubsection{Lengths}
A length is given as a dimension. It may not be omitted.

\subsubsection{Colors}
A color may be one of the names from the definition of {\it color\/} or
some combination of them. A color not in the list of names may be
obtained by ``mixing'' colors (e.g. pink may be specified as
\.{red\&white}). If two graphic objects overlap that are of different
colors, then the object whose definition occurs later in the page should
take precedence.

\subsubsection{Line Cap Shapes}
The shape of the end of a line may be one of three forms:\footnote
{\cite[p.~217]{adobe:psref} contains illustrations of three of the line
caps described}
\begin{description}
\item[\.{butt}] The line is squared off at its endpoint.
\item[\.{round}] A semicircle with radius equal to half of the line width
         is drawn at the end of the line.
\item[\.{square}] The line is extended one half of the line width past
         its endpoint and squared off.
\item[\.{bar}] If the line is vertical at its endpoint, this is treated
         as a butt cap. Otherwise, a vertical line is drawn through the
         endpoint, and the edges of the lines are extended or retracted
         to meet the vertical line drawn.\footnote{This is primarily
         intended for musical typesetting. See \cite{gourlay:music} and
         \cite{renner:textyl} for more information on the subject.}
\end{description}

\subsubsection{File Names}\label{sec:filename}
File names are specified in the same manner as used by \TeX\ and \MF\
(see \cite[Part 29]{knuth:texprog} for details and \.{WEB} code).
The default directory specification is left up to the \DVI\ driver
designer, although it is suggested that there be some standard
directory defined for commonly used included images (such as a
letterhead). Default file extensions are dependent on the context
of {\it file\_name.\/}
%%%BC I guess that I end up with the same thoughts that I had back on
%%%BC   the tags. The extension should be furnished by the driver
%%%BC   because that becomes system dependent, not DVI.


\section{Environment setting commands}\label{sec:env}
The commands described here are used to set certain parameters on
graphics, etc., %%%BC what is this?
                %%% changed \&c to etc.
    \Special\ commands. They only affect %%%BC changed effect to affect
    the {\it cmd\/}
following them. For example, In the following series of \Special\
commands, the \.{color} special will only affect the \.{line} command,
not the
\.{arc}.
\begin{verbatim}
\special{color red line "0 1in; arc "-10 "10 1in}

\special{color red line "0 1in}
\special{arc "-10 "10 1in}
\end{verbatim}
%%%BC The above is questionable! It is enough for the user to
%%%BC   have to understand TeX's difficulties without having
%%%BC   to learn arbitrariness that may be in the driver too.
%%%BC Leave as is for now!

The %%%BC removed an extra word
%%%BC commands
    individual commands are described below, organized by their
effects. %%%BC added an `s'

\subsection{Color setting commands}
The \.{black} and \.{white} commands set colors to be used by graphics
commands.
\begin{description}
\item[\.{black} {\it color\/}] Set the color to be used for ``black''.
         {\it i.e.,\/} the color that lines and curves of graphics
         primitives are to be drawn in and the color to be used for
         pixels to be colored in an included graphic image. The default
         black is \.{black}.
\item[\.{white} {\it color\/}] Set the color to be used for ``white''.
         {\it i.e.,\/} the color that will be used for pixels that
         are set off in an included graphic image. The default white is
         \.{white}.
\end{description}

\subsection{Line and curve parameter commands}
These commands are intended to affect the shapes of lines and curves
drawn by graphics primitives.
\begin{description}
\item[\.{linecap} {\it cap\_shape}] Set the shape of the ends of lines.
         See the description of {\it cap\_shape\/} above for details
         on that parameter. The default line cap is \.{butt}.
\item[\.{linewidth} {\it length}] Set the width of lines and curves to be
         drawn by graphics primitives. The edges of the line will be
         half the line width from the zero-width path of the curve or
         line. The default line width is $0.38889$pt.
\end{description}

\subsection{Translation and rotation}
Graphic output is normally placed at the current \DVI\ $h,v$ coordinates.
These commands alter the placement and sizing of graphic output from a
\Special\ command.
\begin{description}
\item[\.{offset} {\it position\/}] Move the output of the \Special\
         command by/to {\it position.\/} The default is to print it
         where it appears.
\item[\.{turn} {\it angle\/}] Rotate the output of the \Special\
         command about the position where it appears. The default is
         to leave it unrotated.
\item[\.{scale} {\it scaling\/}] Scale the output of the \Special\
         command by the amount specified by {\it scaling.\/} The default
         is to print the output at the natural size.
\end{description}

\section{Proposed  commands}\label{sec:cmds}
%%%BC No {\it cmd\/} may not have an affect on any page than the one upon
%%%BC suggested new line for above
No {\it cmd\/} may affect any page other than the one upon
which it appears. A {\it cmd\/} also may not change the positioning
of text. The commands proposed are listed below, organized by type.
%%%BC I am sure that there is reason for the suggestion. However,
%%%BC   I would sure want the next one, DUPLEX, to be global.
%%%BC What commands did you have in mind?
%%%DH There is no duplex special; duplex printing should be specified by
%%%DH a command line option or some such.

\subsection{Duplex printing}
Duplex printing ({\it i.e.,\/} printing on both sides of a sheet of
paper, is a feature available with some output devices. Duplex printing
can only be selected on a global basis (generally through a command-line
option), but a new sheet may be forced if necessary (for example, for
a table of contents page that is output at the end of the document).
This is accomplished with the \.{newsheet} command.
\begin{description}
\item[\.{newsheet}] Force a new sheet on output. If \.{newsheet} appears
         on a page that would normally be printed on the back of a sheet
         of paper than that page will be printed on a new sheet of paper.
         If the \Special\ appears on the front of a sheet of paper (this
         is always the case in simplex printing) then \.{newsheet} has no
         effect.
\end{description}

\subsection{Page orientation}
Most output devices have the ability to print output rotated
by~$90^\circ$ and sometimes other values. This can be globally specified
{\sl only\/} by the user interface to the \DVI-driver. Changes on a {\sl
single page basis\/} may be made through \Special\ commands listed below
(These commands will generally be issued inside of a command pair
%%%BC added pair to the above and endinsert below 2 lines
    such as
\verb|\pageinsert ... \endinsert|).
\begin{description}
\item[\.{rotate} {\it angle\/}] The page that \.{rotate} appears upon
         will be printed rotated $n^\circ$ counterclockwise from the
         vertical position on the actual sheet. If necessary, the
         angle should be rounded as outlined in section~\ref{sec:angle}.
         If more than one \.{rotate} occurs on a given page, the one
         occurring later takes precedence.
\item{\.{landscape}} A specialized case of \.{rotate}. {\it angle\/} is
         set to $90^\circ$.\footnote{Or $-90^\circ$ if $90^\circ$ is not
         available and the former is.}
\item{\.{portrait}} A specialized case of \.{rotate}. {\it angle\/} is
         set to $0^\circ$. This is likely to only be of use in a document
         with a global orientation other than $0^\circ$.
\end{description}

\subsection{Graphics Inclusion}
To maintain a reasonable degree of compatibility between device drivers
for graphics inclusion, a graphics inclusion facility must know something
about
%%%BC removed ->      both
           the nature of the graphics file. This may be either the
intended destination of the graphics file (for example, a file may
be composed of Hewlett-Packard LaserJet Plus control codes) or the
source of the file (MacPaint, for example can generate \PS\ output;
however, it contains additional header information which must be stripped
off of the file before it can be included in a \PS\ file generated by
some \DVI-to-PS program).

A \DVI\ driver should insure that the current {\em real\/} values of
$h$~and~$v$ are not changed by any included graphics. If the device
allows absolute positioning on the page, then the simplest approach
is to simply reset the position on the page; otherwise the driver
{\em must\/} keep track of position changes in the included file
and compensate appropriately afterwards.

\begin{description}
\item[\.{include} {\it tag\/} {\it file\_name}] Include the graphics
         file in {\it file\_name\/} of type {\it tag.\/}

         {\it tag\/} indicates the nature of the graphics file.
         Intelligent drivers should be able to interpret ``foreign''
         graphics formats. Possible values for {\it tag\/} are: (this is
         not a full list; other values may be added as necessary to
         reflect new printers or graphics software)
         \begin{description}
         \item[\.{BITMAP:}] The file is a bitmap of an image with the
                   first fullword indicating the width of the bitmap
                   (in pixels), the second fullword indicating the
                   height of the bitmap, and third fullword indicating
                   the intended resolution (in dpi). The data following
                   this is the binary data for the bitmap scanning from
                   left to right horizontally and top to bottom
                   vertically. If rows will not align on fullwords,
                   then they will be padded on the right with 0~bits.
         \item[\.{EPIC:}] The file is composed of EPIC graphics commands.
         \item[\.{HALFTONE:}] The file contains a halftone graphics
                   image. Format is similar to that for \.{BITMAP:}
                   except that the first byte contains a value indicating
                   how many bits each pixel requires to store its
                   gray-scale value.
         \item[\.{IMP:}] The file is composed of \.{IMPRESS} commands.
         \item[\.{INTER:}] The file is composed of Xerox Interpress
                   commands.
         \item[\.{MAC:}] The file is a \PS\ file created by a MacIntosh
                   graphics program such as MacPaint.
         \item[\.{PCL:}] The file is composed of PCL instructions.
         \item[\.{PS:}] The file is composed of \PS\ commands.
         \item[\.{QUIC:}] The file is composed of {\sf QUIC} control
                   sequences.
         \item[\.{TEK:}] The file is composed of Tektronics plot
                   commands.
         \end{description}

         The file name is as described above in
         section~\ref{sec:filename}. The default filetype is {\it tag\/}
         (less the colon).
\end{description}

\subsection{Graphics primitives}
The ability to generate simple figures using graphics primitives has
been used in the program \.{DVIIMP} \cite{ware:dviimp} and a few
versions of other \DVI-drivers. There is also a \Unix\ utility called
\.{tpic} which uses \.{pic} commands to generate \Special\ commands for
\TeX. The philosophy of these graphics primitives could also be
applied to the \LaTeX\ \.{picture} environment (see \cite{lamport:latex})
to allow more freedom in the choice of graphic objects.

All graphics primitives ignore \.{white}.

The following graphics primitives are provided:
\begin{description}
\item[\.{arc} {\it length\/} {\it angle\/} {\it angle\/}] Draw an arc
         centered at the current position with radius {\it length\/}
         from the first {\it angle\/} counterclockwise to the second
         angle.
\item[\.{circle} {\it length\/}] Draw a circle with radius {\it length\/}
         centered at the current position.
\item[\.{disk} {\it length\/}] Draw a filled circle with radius
         {\it length\/} centered at the current position.
\item[\.{line} {\it length\/} {\it angle}] Draw a line of length
         {\it length\/} at an angle of {\it angle\/} from the current
         point.
\item[\.{pie} {\it length\/} {\it angle\/} {\it angle\/}] Draw two lines
         of length {\it length\/} at the two angles specified, and an
         arc with the {\it length\/} and angles above, and fill the
         figure in with the current \.{black} color.

         Line joins are {\sl mitred.\/} The outer edges of the two lines
         are extended until they meet. If the difference between the two
         angles is less than $10^\circ$ then the join is drawn as if the
         two lines each had round line caps and no special joining was
         done.

         The \.{linecap} setting is ignored.
\item[\.{spline} {\it position\/} {\it position\/} {\it position\/}] A
         B\'ezier cubic is drawn from from the current position to
         the last {\it position\/} in {\it options\/} using the second
         and third {\it position\/}s as control points.
\end{description}

\subsection{Miscellaneous \Special\ commands}
The following miscellaneous commands should also be provided.
\begin{description}
\item[\.{beginbar} {\it tag\/}] Begin a vertical line at the current
         \DVI\ $h,v$ coordinates. The line continues until the $v$
         coordinate of the matching \.{endbar} on the page. If there is
         no matching \.{endbar}, then no line is drawn.
\item[\.{endbar} {\it tag\/}] End a vertical line at the current $v$
         coordinate (see \.{beginbar}).
\item[\.{message} {\it option}] Output {\it option\/} to the terminal.
         If the \DVI-driver does not output page numbers as it processes
         the file, it should print the \TeX\ page number (from the {\it
         bop\/} command see~\cite{ware:dvitype}) before printing {\it
         message\_text.}
\item[\.{textcolor} {\it color}] All text from the first \DVI\ {\it
         push\/} to its matching {\it pop\/} are to be typeset in
         {\it color.\/}
\end{description}

%\bibliography{texref,misc}
%\bibliographystyle{plain}

%%% HOSEK1.BBL
\begin{thebibliography}{10}

\bibitem{aho:compilers}
Alfred~V. Aho, Ravi Sethi, and Jeffrey~D. Ullman.
\newblock {\em Compilers: Principles, Techniques, and Tools}.
\newblock Addison-Wesley, Reading, Massachussetts, 1986.

\bibitem{fox:changebars}
Jim Fox.
\newblock Change bars.
\newblock {\em TUGboat}, 7(3), 1986.
\newblock Appeared in ``Queries'' section.

\bibitem{gourlay:music}
John~S. Gourlay, Allen Parrish, Dean~K. Roush, F.~Javier Sola, and Yiling Tien.
\newblock Computer formatting of music.
\newblock Technical Report OSU-CISRC-2/87-TR3, Ohio State University, 1986.

\bibitem{hosek:specials}
Don Hosek.
\newblock Call for standardization of {\tt dvi} specials.
\newblock {\em \TeXMaG}, 1(5), 1987.

\bibitem{adobe:psref}
Adobe~Systems Incorporated.
\newblock {\em PostScript language reference manual}.
\newblock Addison-Wesley, Reading, Massachussetts, 1985.

\bibitem{ware:dvitype}
Donald~E. Knuth.
\newblock The {\tt dvitype} processor.
\newblock {\tt WEB} program, Stanford University, August 1984.

\bibitem{knuth:texbook}
Donald~E. Knuth.
\newblock {\em The \TeX book}.
\newblock Addison Wesley, Reading, Massachusetts, 1986.

\bibitem{knuth:texprog}
Donald~E. Knuth.
\newblock {\em \TeX: The Program}.
\newblock Addison Wesley, Reading, Massachusetts, 1986.

\bibitem{lamport:latex}
Leslie Lamport.
\newblock {\em \LaTeX: A Document Preparation System}.
\newblock Addison Wesley, Reading, Massachusetts, 1986.

\bibitem{renner:textyl}
John~S. Renner.
\newblock {\TeX}tyl: a line-drawing interface for {\TeX}.
\newblock Technical Report OSU-CISRC-4/87-TR9, Ohio State University, March
  1987.

\bibitem{ware:dviimp}
Arthur~L. Samuel.
\newblock The {\tt dviimp} processor.
\newblock {\tt WEB} program, Stanford University, July 1985.

\end{thebibliography}
\end{document}
