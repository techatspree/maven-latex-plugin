

%\listfiles
\synctex=1% maybe security issue: draft only? 

% for buildParams to check empty: \ifdefempty
%\usepackage{etoolbox}
% for buildParams: \verbdef 
%\usepackage{newverbs}

% provdies \ifPDFTeX, \ifXeTeX and \ifLuaTeX. 
% iftutex test is true for XeTeX and LuaTeX, 
% and an ifpdf test is provided to test the PDF or DVI output mode.
\usepackage{iftex}

% provides \newboolean, \setboolean 
% is used to integrate html production with tex4ht and pdf production
% used to define texFhtLoaded and beamerLoaded 
% maybe this is not really absolute necessary 

\usepackage{ifthen}
% \newboolean{texFhtLoaded}
% \setboolean{texFhtLoaded}{false}

% only with pdflatex, warnings for xelatex and for lualatex 
% ifxetex, ifluatex, ifpdf
\ifpdf%
  %\usepackage{mlmodern}
\else
  %\makeatletter
  \IfPackageLoadedTF{tex4ht}{%
    \setboolean{texFhtLoaded}{true}
  }{%
  }% tex4ht not loaded 
  %\makeatother
\fi


% \newboolean{beamerLoaded}
% %\makeatletter
% \IfClassLoadedTF{beamer}{%
%   \setboolean{beamerLoaded}{true}
% }{
%   \setboolean{beamerLoaded}{false}
% }
% %\makeatother



\iftutex%
  \usepackage{fontspec}
\else
  % this seems to work with beamer also 
  \usepackage[utf8]{inputenc}
  \usepackage[T1]{fontenc}
\fi
%\usepackage{textalpha}


% absolutely necessary. 
% for document development add certain options. 
% Then remove headline and prevent this plugin from overwriting. 

\IfClassLoadedTF{beamer}{
  % here nothing to do. 
  % beamer loads geometry itself. 
  % The option a4paper does not make sense; 
  % one may set aspectratio in \documentclass
}{
  \usepackage[a4paper]{geometry}% option , showframe, showcrop 
}
%\usepackage{showframe} as an alternative 
\usepackage{microtype}
%\usepackage[indent,skip=0]{parskip}% used by pandoc but not good 
% special characters
\usepackage{textcomp}
\usepackage{anyfontsize}% important e.g. for beamer class 
%\usepackage{cleveref}


% used by hyperref and also to update index and glossary 
% to avoid clash because of loading with different options: 
% declare first 
% Note that without options the check is the most strict one 
\usepackage{rerunfilecheck}

% graphics 

\ifpdf%
  % for accessability with luatex
  %\usepackage{luatex85}
  % compiles for xelatex only 
  %\usepackage[tagged, highstructure]{accessibility}
  \usepackage{xcolor}  % [pdftex]  
  \usepackage{graphicx}% [pdftex] 
  % driver [hpdftex] is autodetected 
  \usepackage[destlabel]{hyperref}
  % sometimes comes in with svg import 
  \usepackage{transparent}
  % warning transparent package: 
  % loading aborted if not pdf-mode 
  % strange: according to documentation not for xelatex; 
  % seems to work anyway 
  % can be extended using l3opacity
\else
  % No PDF, includes dvi/xdv and HTML,... via package tex4ht 
  \usepackage[dvipdfmx]{xcolor}
  \usepackage[dvipdfmx]{graphicx}
  \IfPackageLoadedTF{tex4ht}{%
    \usepackage[tex4ht, destlabel]{hyperref}
  }{%
    \ifxetex%
      \usepackage[destlabel]{hyperref}
    \else
      \usepackage[dvipdfmx, destlabel]{hyperref}%[dvipdfmx]
      % lualatex: without [dvipdfmx] option did not find 
      % converter dvi to pdf or to ps
      % pdflatex: without [dvipdfmx] option dvips still works, 
      % but no converter for pdf
    \fi
  }% tex4ht not loaded 
  %\usepackage{bmpsize}% not for xelatex 
\fi%ifpdf

\ifluatex%
  \usepackage{luamplib}
  \newcommand*\inputmpcode[1]{\begin{mplibcode}input #1\end{mplibcode}}
\else
\fi

% \@ifpackageloaded{tex4ht}{%
% \usepackage[dvipdfmx]{xcolor}
% \usepackage[dvipdfmx]{graphicx}
% \usepackage[tex4ht]{hyperref}
% \usepackage{bmpsize}
% }{%
% \usepackage{xcolor}  % [pdftex]  
% \usepackage{graphicx}% [pdftex] 
% \usepackage{hyperref}% driver [hpdftex] is autodetected 
% }


%\usepackage[clear,pdf,eps]{svg}

\usepackage{import}
\usepackage{amsmath}

% synchronization between tex and pdf 
%\usepackage[active]{srcltx}
\usepackage{longtable}
\usepackage{listings}
% this is a workaround for including listings with latexmk.. 
% This can be fixed 
% - as shown below 
% - patch in package listings 
% - patch in latexmk 
% I would prefer the latter. 
\usepackage{xpatch}
\makeatletter
\newcommand*{\NewLine}{^^J}%
\xpatchcmd{\lst@MissingFileError}
{Package Listings Error: File `#1(.#2)' not found.}
{LaTeX Error: File `#1.#2' not found.\NewLine}{%
  \typeout{File ending patch for \string\lst@MissingFileError\space done.}%
}{%
  \typeout{File ending patch for \string\lst@MissingFileError\space failed.}%
}
\makeatother

\usepackage{fancyvrb}


% index and glossary
\IfPackageLoadedTF{tex4ht}{
  \newcommand{\pkg}[1]{\texttt{#1}}% without indexing 
}{
  \usepackage{splitidx}%[split]
%  \usepackage{makeidx}
%  \usepackage{showidx}
  \makeindex
  \usepackage[toc]{glossaries}%,automake
  % , xindy or even [xindy={language=english,codepage=utf8}]
  % mainly for index and glossaries 
  %\makeglossaries% TBD: activate later
  \newcommand{\pkg}[1]{\texttt{#1}\sindex[pkg]{#1}} % TBD: this must be extracted 
  }

% high quality tables 
\usepackage{booktabs}
\aboverulesep=0ex
\belowrulesep=0ex

\usepackage{xurl}

%\makeglossary% for rerunfilecheck 

%\usepackage{etexcmds} %still later 
\IfClassLoadedTF{beamer}{
  % TBD: clarify this case. 
  % maybe beamer does not support indices or glossaries. 
  % 
}{
  \usepackage[nottoc, numindex, numbib]{tocbibind}
}

%\usepackage{latex-bnf}





\ifpdf%
  \ifLuaTeX%
    % for lualatex
    \pdfvariable minorversion=7% chktex 1
    % omit CreationDate and ModDate keys.
    \pdfvariable suppressoptionalinfo 767% chktex 1
    % no adding to the trailer dictionary.
    \pdfvariable trailerid{}% chktex 1
    \pdfvariable suppressoptionalinfo -1% chktex 1
  \else
    \ifXeTeX%
      % for xelatex
      \special{pdf:minorversion 7}
      % TBD: find a way to express pdfinfoomitdate: necessary? 
      \special{pdf:trailerid []}
    \else
      \ifPDFTeX%
        \pdfminorversion=7         % for pdflatex
        % omit CreationDate and ModDate keys.
        % not before pdfTeX 3.14159265-2.6-1.40.17
        \pdfinfoomitdate=1                   % for pdflatex
        % no adding to the trailer dictionary.
        %\pdftrailer=0                        % for pdflatex
        \pdftrailerid{}                       % for pdflatex
        \pdfsuppressptexinfo=-1               % for pdflatex
      \else
        % Here, the tex processor is unknown. 
      \fi%pdftex
    \fi%xetex
  \fi%luatex

  \hypersetup{
    pdfinfo={
      Author      ={Ernst Reissner},
      Title       ={Presentation with/of the latex-maven-plugin },
      CreationDate={unknown},
      ModDate     ={unknown},
      Producer    ={unknown},
      Subject     ={presentations with beamer},
      Keywords    ={LaTeX;beamer}
    }
  }
\else
\fi%ifpdf

% TBD: clarify what to do if no pdf is created. 
% As soon as this question comes up in the course of new development, 
% this will be compiled as html by accident. 
% The we can see, how well this works. 

\usetheme{Berlin}
\title{Presentation with/of the \texttt{latex-maven-plugin}}
\author{E. Reissner}
\date{ernst.reissner@simuline.eu}


\hypersetup{colorlinks,linkcolor=,urlcolor=blue,citecolor=yellow}%
%\hypersetup{frenchlinks}
\begin{document}
 
\mode<article>{\maketitle}




\begin{frame}
  \titlepage%
\end{frame}

\section*{Outline}

\begin{frame}
  \tableofcontents
\end{frame}

\section{Introduction and Purpose}

This is some additional text 

\begin{frame}
  \frametitle{Purpose of these documents }
  The purpose of this document is twofold:
  %
  \begin{itemize}
    \item Give an overview over the plugin. 
    \alert{But: }
    %
    \begin{itemize}
      \item The single official description of the plugin is the manual~\cite{LatexPlugin}.
      \item The 
      \href{http://simuline.eu/LatexMavenPlugin/index.html}{project site} 
      gives already an overview. 
      \item 
      The content of this presentation is updated only by need. 
    \end{itemize}
    
    \item Demonstrate that the plugin can compile a presentation using the \texttt{beamer} class 
    and a handout written as \texttt{beamerarticle}. 

    \begin{itemize}
      \item In fact, when citing this presentation as~\cite{PresBeamer}, 
      this refers to both. 
      \item The presentation is about the plugin 
      \item The handout adds information on how to write a beamer presentation. 
      \item Both documents turn \texttt{beamer} class and package \texttt{beamerarticle} into preferred usage 
      ensuring tests. 
    \end{itemize}
  \end{itemize}
  
\end{frame}


\section{Features}

\subsection{Realized Features}

\begin{frame}
  \frametitle{Ant and Maven}
  Automatically creates documents from LaTeX sources during the Maven \texttt{site} phase 
  and in an ant run. 

  This comprises running basic tools like \texttt{lualatex} and \texttt{bibtex} 
  and rerunning them by need. 

\end{frame}

\begin{frame}
  \frametitle{Supported IO}
  Supports 
  \begin{itemize}
    \item
    many output formats like PDF, DVI, HTML, DOCX, RTF, TXT and others
    \item
    many graphical input formats like PNG, MP, FIG, gnuplot; 
    also provides a separate goal creating them, \texttt{grp} 
    \item
    bibliography, index, glossary and embedded code; in particular split index
  \end{itemize}
\end{frame}


\begin{frame}
  \frametitle{Checks}
  Check 
  %
  \begin{itemize}
    \item
    sources with \texttt{chktex} and logs the results in target and goal \texttt{chk} 
    \item
    versions of used tools via goal \texttt{vrs} 
    \item
    log files detecting errors and warnings 
    \item
    whether a document could have been reproduced, by demand 
  \end{itemize}
\end{frame}

\begin{frame}
  \frametitle{Orchestration and document development}
  Orchestration of various tools detecting need for execution e.g.
  of \texttt{bibtex} including \texttt{rerunfilecheck} for \texttt{lualatex} and friends.

  Supports document development, mainly by cooperating with editor, viewer and with other tools in the build chain.
  \begin{itemize}
    \item Offers installation script for extensions of VS Code.
    \item Offers configuration file \texttt{.chktexrc} for \texttt{chktex}.
    \item Can create configuration file \texttt{.latexmkrc} 
    for \texttt{latexmk} synchronized with the configuration.
    \item Offers a common header file \texttt{header.tex} 
    to unify packages loaded by latex main files.
  \end{itemize}
\end{frame}


\subsection{Planned Features}

\begin{frame}
  \frametitle{Planned Features}
  \begin{itemize}
  \item Support \texttt{biber} replacing \texttt{bibtex} as preferred tool
  \item Support \texttt{xindy} replacing \texttt{makeindex} as preferred tool
  \item Support \texttt{bib2gls} replacing \texttt{makeglossary} as preferred tool
  \item Execute \texttt{glosstex} if needed
  \item Usage of the \texttt{multibib} macros
  \item \dots
  \end{itemize}
\end{frame}

% \section{Installation}

\section{Goals}

\begin{frame}
  \frametitle{Goals}

  An overview is on the 
  \href{http://simuline.eu/LatexMavenPlugin/plugin-info.html}{goals page}. 
  
  \setbeamercolor{alerted text}{fg=blue}
  Besides goals referring to creating output of a given type 
  like \alert{\texttt{pdf}}, \alert{\texttt{odt}} and \alert{\texttt{html}}, 
  there is a goal \alert{\texttt{cfg}}, 
  which allows \setbeamercolor{alerted text}{fg=red}
  \alert con\alert fi\alert guring various output formats as \texttt{targets}. % chktex 1
  
  \setbeamercolor{alerted text}{fg=blue}
  One further goal, \alert{\texttt{chk}} to perform a check with \texttt{chktex} 
  is allowed among the \texttt{targets}. 

  The other goals cannot serve as targets: 
  %
  \begin{description}
    \item[\texttt{vrs}] to check versions of tools, 
    \item[\texttt{grp}] to create graphic files speeding up use of \texttt{latexmk} 
    and 
    \item[\texttt{inj}] to inject useful files like headers or config files 
    like \texttt{.latexmkrc}. 
  \end{description}
  

\end{frame}

% \section{Usage}

% \section{Examples}

\mode<presentation>
\section{References}% if with star not in toc 


\begin{frame}[allowframebreaks]
  \frametitle{References}
  \bibliographystyle{alpha}
  \bibliography{../lit}{}
\end{frame}

\mode<article>
\bibliographystyle{alpha}
\bibliography{../lit}{}

\end{document}