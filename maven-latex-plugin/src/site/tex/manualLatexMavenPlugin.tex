
\documentclass[12pt]{article}

\usepackage{longtable}

\title{Manual for the latex-maven-plugin and for an according ant task }
\author{Ernst Reissner (rei3ner@arcor.de)}

\begin{document}
\maketitle

\tableofcontents

\section{Introduction}

This piece of software implements both an ant-task and a maven-plugin 
generating documentation of various formats from LaTeX files 
in an uniform way. 
In particular, the settings which may be passed to the task 
and those allowed for the plugin are in a one-to-one relation. 
They are both described in Section \ref{sec:settings}. 

Both, the ant-task and the maven-plugin rely on the same code base 
which form the package {\tt org.m2latex.core}. 
The code specific for the ant task is in {\tt org.m2latex.antTask} 
and that specific for the maven plugin is in {\tt org.m2latex.mojo}. 

It is very usual to endow LaTeX-files with figures. 
We support figures created in the xfig-format 
by the program xfig. 
Further make functionality on latex figures is not supported, 
but can be easily added. 
If there is some need, please write an email to the author. 

The creation process supports makeindex and bibtex. 
Again, further functionality can be added by demand. 


\section{Settings}\label{sec:settings}

This section describes the parameters 
of both the ant-task and the maven-plugin. 


\begin{longtable}{|ll|}
\hline
Parameter        & Default  \\
\multicolumn2{|l|}{Explanation }  \\
\hline
\hline
\multicolumn2{|l|}{directories, miscellaneous } \\
\hline
\tt texSrcDirectory  & \tt src/site/tex  \\
\multicolumn2{|l|}{
\begin{minipage}{1.0\linewidth}
The tex source directory as a string, 
containing all tex main documents 
(including subfolders) to be processed
relative to {\tt\$baseDirectory}. 
The default value is '{\tt src/site/tex}' on Unix systems. 
\end{minipage}
} \\
\tt tempDirectory    & \tt m2latex       \\
\multicolumn2{|l|}{
\begin{minipage}{1.0\linewidth}
The working directory, 
for temporary files and LaTeX processing 
relative to {\tt \$targetDirectory} 
which is by default '{\tt\$baseDirectory/target}' on Unix systems. 
%The default value is '{\tt m2latex}'. 
\end{minipage}
} \\
\tt outputDirectory  & \tt .             \\
\multicolumn2{|l|}{
\begin{minipage}{1.0\linewidth}
The generated artifacts will be copied to this folder 
relative to {\tt\$targetSiteDirectory} 
which is by default '{\tt\$targetDirectory/site}' on Unix systems. 
%The default value is '{\tt.}'.  
\end{minipage}
} \\
\tt targets          & \tt pdf, html     \\
\multicolumn2{|l|}{
\begin{minipage}{1.0\linewidth}
A comma separated list of targets to be stored in {\tt\$targetSet}. 
%The default value is '{\tt pdf, html}'. 
\end{minipage}
} \\
\tt texPath          & {\tt null}         \\
\multicolumn2{|l|}{
\begin{minipage}{1.0\linewidth}
Path to the TeX scripts or {\tt null}. 
In the latter case, the scripts must be on the system path. 
Note that in the pom, {\tt$<$texPath/$>$} 
and even {\tt$<$texPath$>$    $<$/texPath$>$} represent the {\tt null}-File. 
%The default value is {\tt null}. 
\end{minipage}
} \\
\tt fig2devCommand   & \tt fig2dev       \\
\multicolumn2{|l|}{
\begin{minipage}{1.0\linewidth}
The fig2dev command for conversion of fig-files into various formats. 
Currently only {\tt pdf} combined with {\tt pdf\_t} is supported. 
%The default value is '{\tt fig2dev}'. 
\end{minipage}
} \\
\hline
\multicolumn2{|l|}{latex to pdf} \\
\hline
\tt texCommand       & \tt pdflatex      \\
\multicolumn2{|l|}{
\begin{minipage}{1.0\linewidth}
The LaTeX command to create a pdf-file. 
%The default value is '{\tt pdflatex}'. 
\end{minipage}
} \\
\tt texCommandArgs   & \tt\small -interaction=nonstopmode -src-specials  \\
\multicolumn2{|l|}{
\begin{minipage}{1.0\linewidth}
The arguments string to use 
when calling LaTeX via {\tt\$texCommand}.
Leading and trailing blanks are ignored. 
A sequence of at least one blank separate the proper options. 
%The default value is 
%'{\tt-interaction=nonstopmode -src-specials}'. 
\end{minipage}
} \\
\tt patternerrlatex & \tt\small! $|$Fatal error$|$LaTeX Error$|$Emergency stop\\
\multicolumn2{|l|}{
\begin{minipage}{1.0\linewidth}
The pattern in the log-file 
indicating a failure when running the {\tt\$texCommand}. 
%The default value is 
%'{\tt ! $|$Fatal error$|$LaTeX Error$|$Emergency stop}'. 
If this is not complete, please extend 
and notify the developer of this plugin. 
\end{minipage}
} \\
\tt debugbadboxes    & \tt true          \\
\multicolumn2{|l|}{
\begin{minipage}{1.0\linewidth}
Whether debugging of overfull/underfull hboxes/vboxes is on: 
If so, a bad box occurs in the last LaTeX run, a warning is displayed. 
For details, set {\tt\$cleanUp} to false, 
rerun LaTeX and have a look at the log-file.
%The default value is '{\tt true}'. 
\end{minipage}
} \\
\tt debugwarnings    & \tt true          \\
\multicolumn2{|l|}{
\begin{minipage}{1.0\linewidth}
Whether debugging of warnings is on: 
If so, a warning in the last LaTeX run is displayed. 
For details, set {\tt\$cleanUp} to false, 
rerun LaTeX and have a look at the log-file. 
%The default value is '{\tt true}'. 
\end{minipage}
} \\
\hline
\multicolumn2{|l|}{bibtex and makeindex} \\
\hline
\tt bibtexcommand    & \tt bibtex        \\
\multicolumn2{|l|}{
\begin{minipage}{1.0\linewidth}
The BibTeX command to create a bbl-file 
from an aux-file and a bib-file (using a bst-style file). 
%The default value is '{\tt bibtex}'. 
\end{minipage}
} \\
\tt makeindexcommand & \tt makeindex     \\
\multicolumn2{|l|}{
\begin{minipage}{1.0\linewidth}
The MakeIndex command to create an ind-file from an idx-file 
logging on an ilg-file. 
%The default value is '{\tt makeindex}'. 
\end{minipage}
} \\
\tt patternErrMakeindex & see below ****           \\
\multicolumn2{|l|}{
\begin{minipage}{1.0\linewidth}
The pattern in the ilg-file 
indicating that \$makeIndexCommand failed. 
The default value is chosen 
according to the 'makeindex' documentation 
but seems to be incomplete.
If this is not complete, please extend 
and notify the developer of this plugin. \end{minipage}
} \\
\hline
\multicolumn2{|l|}{htlatex} \\
\hline
\tt tex4htCommand       & \tt htlatex  \\
\multicolumn2{|l|}{
\begin{minipage}{1.0\linewidth}
\end{minipage}
} \\
\tt tex4htStyOptions    & \tt xhtml,uni-html4,2,svg  \\
\multicolumn2{|l|}{
\begin{minipage}{1.0\linewidth}
\end{minipage}
} \\
\tt tex4htOptions       & \tt -cunihtf -utf8         \\
\multicolumn2{|l|}{
\begin{minipage}{1.0\linewidth}
\end{minipage}
} \\
\tt t4htOptions         & the empty string              \\%-cvalidate
\multicolumn2{|l|}{
\begin{minipage}{1.0\linewidth}
The options for 't4ht' which converts idv-file and lg-file 
into css-files, tmp-file and, 
by need and if configured accordingly into png files. 
The value '-p' prevents creation of png-pictures.
%The default value is the empty string. 
\end{minipage}
} \\
\tt patternNeedLatexReRun &  see below ****          \\
\multicolumn2{|l|}{
\begin{minipage}{1.0\linewidth}
The pattern in the log file which triggers rerunning latex. 
This pattern may never be ensured to be complete, 
because any new package may break completeness. 
Nevertheless, the default value aims completeness 
while be restrictive enough not to trigger another latex run if not needed. 
To ensure termination, let \$maxNumReruns 
specify the maximum number of latex runs. 
If the user finds an extension, (s)he is asked to contribute 
and to notify the developer of this plugin. 
Then the default value will be extended. 
\end{minipage}
} \\
\tt maxNumReruns        & \tt 5               \\
\multicolumn2{|l|}{
\begin{minipage}{1.0\linewidth}
The maximal allowed number of reruns of the latex process. 
This is to avoid endless repetitions. 
%The default value is 5. 
This shall be non-negative 
or -1 which signifies that there is no threshold. 
\end{minipage}
} \\
\tt latex2rtfCommand    & \tt latex2rtf        \\
\multicolumn2{|l|}{
\begin{minipage}{1.0\linewidth}
The latex2rtf command to create rtf from latex directly. 
%The default value is '{\tt latex2rtf}'. 
\end{minipage}
} \\
\tt odt2docCommand      & \tt odt2doc          \\
\multicolumn2{|l|}{
\begin{minipage}{1.0\linewidth}
The odt2doc command to create MS word-formats from otd-files. 
%The default value is '{\tt odt2doc}'. 
\end{minipage}
} \\
\tt odt2docOptions      & \tt -fdocx          \\
\multicolumn2{|l|}{
\begin{minipage}{1.0\linewidth}
The options of the odt2doc command. 
Above all specification of output format via the option '-f'. 
The odt2doc command is invoked in the form 
'{\tt odt2doc -f$<$format$>$ $<$file$>$.odt}'. 
All output formats are shown by '{\tt odt2doc --show}' 
but the formats interesting in this context 
are {\tt doc}, {\tt doc6}, {\tt doc95}, {\tt docbook}, {\tt docx}, 
{\tt docx7}, {\tt ooxml}, {\tt rtf}. 
Interesting also the verbosity options '{\tt -v}', '{\tt -vv}', '{\tt -vvv}' 
the timeout '{\tt -T=secs}' and '{\tt --preserve}' 
to keep permissions and timestamp of the original document. 
%The default value is '{\tt -fdocx}'. 
\end{minipage}
} \\
\tt pdf2txtCommand      & \tt pdftotext        \\
\multicolumn2{|l|}{
\begin{minipage}{1.0\linewidth}
The pdf2txt command converting pdf into plain text. 
The default value is '{\tt pdftotext}'. 
\end{minipage}
} \\
\tt pdf2txtOptions      & the empty string  \\
\multicolumn2{|l|}{
\begin{minipage}{1.0\linewidth}
The options of the pdf2txt command. 
%The default value is the empty string. 
\end{minipage}
} \\
\tt cleanUp             & \tt true             \\
\multicolumn2{|l|}{
\begin{minipage}{1.0\linewidth}
Clean up the working directory in the end? 
May be used for debugging when setting {\tt false}. 
%The default value is '{\tt true}'. 
\end{minipage}
} \\
\hline
\caption{\label{tab:parameters} The parameters of the ant-task 
  and of the maven-plugin. }
\end{longtable}



\section{Goals}

%\begin{longtable}

%\end{longtable}

\section{Installation}

The ant task is tested with 
{\tt Apache Ant(TM) version 1.9.4 compiled on September 11 2015}
and the maven plugin with 
%
\begin{verbatim}
Apache Maven 3.2.1
(ea8b2b07643dbb1b84b6d16e1f08391b666bc1e9; 
2014-02-14T18:37:52+01:00). 
\end{verbatim}
The java version is {\tt 1.8.0\_101, vendor: Oracle Corporation}. 

The task just passes parameters in the build file to the core 
and accordingly the maven plugin passes parameters in the pom 
to the core. 
The core just invokes various programs to do the actual work. 

To process the fig-files, by default {\tt fig2dev} is used. 
To create pdf-files from latex files we use {\tt pdflatex} 
or some other kind of latex creating pdf-files. 
LaTeX uses several auxiliary programs. 
Above all {\tt bibtex} to create the bibliography 
and {\tt makeindex} for the index. 

To create html-files, or to be more precise any kind of sgml and xml, 
from pdf-files, {\tt htlatex} is used. 
This includes also creating open office documents like odt-files. 
Thus open office documents are created in two steps, 
the first is to create pdf-files with the according tools, 
the second one is done by {\tt htlatex}. 

To create rtf-files, currently {\tt latex2rtf} is used. 
Note that this does not require {\tt pdflatex}. 
As a drawback, not all latex-packages are supported. 

MS word documents are created from open office documents via {\tt odt2doc} 
and thus require three steps. 

Finally, there is a way, to create plain text files from the pdf-files 
via {\tt pdftotext}. 
The way from latex to text via pdf makes sense 
because that text is well formatted 
and may contain unicode symbols like $\pi$. 

So to run this software, the abovementioned programs must be installed. 

TODO: how to install the maven-plugin and how the ant-task. 

\section{Gaps}


\end{document}

%%% Local Variables: 
%%% mode: latex
%%% TeX-master: t
%%% End: 
