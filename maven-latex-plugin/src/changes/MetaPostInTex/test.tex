\documentclass{article}
\synctex=1
\usepackage{fontspec}
\usepackage{luamplib}
\newcommand*\inputmpcode[1]{\begin{mplibcode}input #1.mp\end{mplibcode}}

\begin{document}
This works with lualatex only. 
Modified \texttt{/usr/share/texmf/luatex/luamplib/luamplib.cfg} 
to get rid of additional files. 
What I hate on \texttt{luamplib} is that it does not provide the command 
\texttt{\textbackslash{}inputmpcode} which I have to provide myself. 
This command is important to provide standalone metapost files 
and to get access to appropriate code highlighting. 

\newcommand{\mytest}{\texttt{my newest test}}

\inputmpcode{metapostOnMetaUml}
% \begin{mplibcode}
% input metauml;
% beginfig(1);
% draw btex \mytest etex;
%   Note.xxx("Hello");
%   drawObjects(xxx);
%   endfig;
%   end
% bye
% \end{mplibcode}
\end{document}